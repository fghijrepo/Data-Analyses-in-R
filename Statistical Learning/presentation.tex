% Options for packages loaded elsewhere
\PassOptionsToPackage{unicode}{hyperref}
\PassOptionsToPackage{hyphens}{url}
%
\documentclass[
  ignorenonframetext,
]{beamer}
\title{MS Project Presentation}
\subtitle{Bias-Variance Tradeoff and Cross-Validation}
\author{Funfay Jen}
\date{2022-01-03}

\usepackage{pgfpages}
\setbeamertemplate{caption}[numbered]
\setbeamertemplate{caption label separator}{: }
\setbeamercolor{caption name}{fg=normal text.fg}
\beamertemplatenavigationsymbolsempty
% Prevent slide breaks in the middle of a paragraph
\widowpenalties 1 10000
\raggedbottom
\setbeamertemplate{part page}{
  \centering
  \begin{beamercolorbox}[sep=16pt,center]{part title}
    \usebeamerfont{part title}\insertpart\par
  \end{beamercolorbox}
}
\setbeamertemplate{section page}{
  \centering
  \begin{beamercolorbox}[sep=12pt,center]{part title}
    \usebeamerfont{section title}\insertsection\par
  \end{beamercolorbox}
}
\setbeamertemplate{subsection page}{
  \centering
  \begin{beamercolorbox}[sep=8pt,center]{part title}
    \usebeamerfont{subsection title}\insertsubsection\par
  \end{beamercolorbox}
}
\AtBeginPart{
  \frame{\partpage}
}
\AtBeginSection{
  \ifbibliography
  \else
    \frame{\sectionpage}
  \fi
}
\AtBeginSubsection{
  \frame{\subsectionpage}
}
\usepackage{amsmath,amssymb}
\usepackage{lmodern}
\usepackage{iftex}
\ifPDFTeX
  \usepackage[T1]{fontenc}
  \usepackage[utf8]{inputenc}
  \usepackage{textcomp} % provide euro and other symbols
\else % if luatex or xetex
  \usepackage{unicode-math}
  \defaultfontfeatures{Scale=MatchLowercase}
  \defaultfontfeatures[\rmfamily]{Ligatures=TeX,Scale=1}
\fi
\usetheme[]{CambridgeUS}
\usecolortheme{beaver}
\usefonttheme{structurebold}
% Use upquote if available, for straight quotes in verbatim environments
\IfFileExists{upquote.sty}{\usepackage{upquote}}{}
\IfFileExists{microtype.sty}{% use microtype if available
  \usepackage[]{microtype}
  \UseMicrotypeSet[protrusion]{basicmath} % disable protrusion for tt fonts
}{}
\makeatletter
\@ifundefined{KOMAClassName}{% if non-KOMA class
  \IfFileExists{parskip.sty}{%
    \usepackage{parskip}
  }{% else
    \setlength{\parindent}{0pt}
    \setlength{\parskip}{6pt plus 2pt minus 1pt}}
}{% if KOMA class
  \KOMAoptions{parskip=half}}
\makeatother
\usepackage{xcolor}
\IfFileExists{xurl.sty}{\usepackage{xurl}}{} % add URL line breaks if available
\IfFileExists{bookmark.sty}{\usepackage{bookmark}}{\usepackage{hyperref}}
\hypersetup{
  pdftitle={MS Project Presentation},
  pdfauthor={Funfay Jen},
  hidelinks,
  pdfcreator={LaTeX via pandoc}}
\urlstyle{same} % disable monospaced font for URLs
\newif\ifbibliography
\usepackage{longtable,booktabs,array}
\usepackage{calc} % for calculating minipage widths
\usepackage{caption}
% Make caption package work with longtable
\makeatletter
\def\fnum@table{\tablename~\thetable}
\makeatother
\usepackage{graphicx}
\makeatletter
\def\maxwidth{\ifdim\Gin@nat@width>\linewidth\linewidth\else\Gin@nat@width\fi}
\def\maxheight{\ifdim\Gin@nat@height>\textheight\textheight\else\Gin@nat@height\fi}
\makeatother
% Scale images if necessary, so that they will not overflow the page
% margins by default, and it is still possible to overwrite the defaults
% using explicit options in \includegraphics[width, height, ...]{}
\setkeys{Gin}{width=\maxwidth,height=\maxheight,keepaspectratio}
% Set default figure placement to htbp
\makeatletter
\def\fps@figure{htbp}
\makeatother
\setlength{\emergencystretch}{3em} % prevent overfull lines
\providecommand{\tightlist}{%
  \setlength{\itemsep}{0pt}\setlength{\parskip}{0pt}}
\setcounter{secnumdepth}{-\maxdimen} % remove section numbering
\titlegraphic{\centering \includegraphics[width=3cm]{tradeoff.jpg}}
\ifLuaTeX
  \usepackage{selnolig}  % disable illegal ligatures
\fi
\usepackage[]{natbib}
\bibliographystyle{plainnat}

\begin{document}
\frame{\titlepage}

\begin{frame}[allowframebreaks]
  \tableofcontents[hideallsubsections]
\end{frame}
\hypertarget{background-and-motivation}{%
\section{Background and Motivation}\label{background-and-motivation}}

\begin{frame}{What is the Bias-Variance Tradeoff?}
\protect\hypertarget{what-is-the-bias-variance-tradeoff}{}
\begin{itemize}
\tightlist
\item
  A central problem in \textbf{supervised learning}
\item
  Desires a model that simultaneously:

  \begin{itemize}
  \tightlist
  \item
    Captures the regularities in its training data
  \item
    Generalizes well to unseen data
  \end{itemize}
\item
  Unfortunately,

  \begin{itemize}
  \tightlist
  \item
    High-variance learning methods may be able to represent their training set well but are at risk of overfitting to noisy or unrepresentative training data.
  \item
    In contrast, algorithms with high bias typically produce simpler models that may fail to capture important regularities (i.e.~underfit) in the data.
  \end{itemize}
\end{itemize}
\end{frame}

\begin{frame}{Mathematics}
\protect\hypertarget{mathematics}{}
\begin{itemize}
\tightlist
\item
  \(f\): fixed but unknown function
\item
  \(X\): can be univariate or multivariate
\item
  \(Y\): univariate
\item
  \(\epsilon\): zero mean, finite variance
\end{itemize}
\end{frame}

\begin{frame}{Mathematics}
\protect\hypertarget{mathematics-1}{}
\begin{equation}
E_{D, \epsilon} [(Y - \hat f(X; D))^2] = (Bias_D [\hat f(X; D)])^2 + Var_D [\hat f(X; D)]) + \sigma^2 \label{eq:MSE}
\end{equation}

where

\begin{equation}
Bias_D[\hat f(X; D) = E_D[\hat f(X; D)]] - f(X) \label{eq:bias}
\end{equation}

and

\begin{equation}
Var_D [\hat f(X; D)] = E_D[(E_D [\hat f(X; D)] - \hat f(X; D))^2] \label{eq:var}
\end{equation}
\end{frame}

\begin{frame}{Mathematics}
\protect\hypertarget{mathematics-2}{}
Since all three terms are non-negative, the irreducible error forms a lower bound on the expected error on unseen samples.

\begin{itemize}
\tightlist
\item
  \(\sigma^2\): lowest achievable predictive MSE
\item
  bias and variance: adds to it to result in a total MSE
\item
  tuning parameter: not visible in Equations \eqref{eq:MSE}, \eqref{eq:bias}, \eqref{eq:var}, but modulates the relative contributions of these terms
\end{itemize}
\end{frame}

\hypertarget{method-of-simulating-the-bias-and-variance-tradeoff}{%
\section{Method of Simulating the Bias and Variance Tradeoff}\label{method-of-simulating-the-bias-and-variance-tradeoff}}

\begin{frame}{At a high level}
\protect\hypertarget{at-a-high-level}{}
\begin{itemize}
\tightlist
\item
  Visualized as a plot of the total expected mean squared error (MSE) along with its three components (bias, variance and the irreducible error) against one or more hyperparameters.

  \begin{itemize}
  \tightlist
  \item
    Compute the expectations in Equations \eqref{eq:MSE}, \eqref{eq:bias}, and \eqref{eq:var} via Monte Carlo simulation
  \item
    Do this for each set of hyperparameters to trace out a path (or high-dimensional surface)
  \item
    Visualize the resulting bias-variance tradeoff
  \end{itemize}
\end{itemize}
\end{frame}

\begin{frame}{Procedurally}
\protect\hypertarget{procedurally}{}
(Refer to the report for details)

\begin{enumerate}
\item
  Generate Training Sets
\item
  Generate Test Sets
\item
  Deploy the Statistical Method at a given set of hyperparameters
\item
  Compute Bias, Variance and MSE according to Equations \eqref{eq:bias}, \eqref{eq:var}, and \eqref{eq:MSE}
\item
  Repeat steps 1-4 over a grid of (sets of) hyperparameters and visualize the resulting mean squared error profile and how it is decomposed into its components.
\end{enumerate}
\end{frame}

\hypertarget{method-of-cross-validation}{%
\section{Method of Cross-Validation}\label{method-of-cross-validation}}

\begin{frame}{Setup}
\protect\hypertarget{setup}{}
\begin{itemize}
\tightlist
\item
  Generate \emph{one} single set of data containing observations
\item
  Randomly divide the set of observations into \(K\) folds of approximately equal size.
\item
  Each fold \(i\), \(1 \le i \le K\) is treated as a test set and the method is fit on the remaining \(K - 1\) folds. The mean squared error, \(MSE_i\), is computed on the observations in the held-out fold.
\item
  The K-fold CV estimate is computed by averaging these values,
  \begin{equation}
  CV_{(K)} = \frac{1}{K}\sum_{i=1}^K MSE_i \label{eq:cv-mse}
  \end{equation}
\end{itemize}
\end{frame}

\begin{frame}{\# Relationship between the Bias-Variance Tradeoff and Cross-Validation}
\protect\hypertarget{relationship-between-the-bias-variance-tradeoff-and-cross-validation}{}
\includegraphics[width=1\textwidth,height=1\textheight]{./mc_vs_cv.png}
\end{frame}

\begin{frame}{Recap of the Main Differences}
\protect\hypertarget{recap-of-the-main-differences}{}
\begin{itemize}
\tightlist
\item
  With CV, we cannot determine the bias component since in practice we do not know the true underlying function \(f\);
\item
  With CV, we cannot determine the variance component since in practice we often only have one set of data available and the \(K\) folds typically do not have the same \(X\) values--we cannot properly define variance in this setting;
\item
  With CV, we can still compute an average MSE over the folds. However, one issue is that the MSE's on different folds become correlated.
\end{itemize}
\end{frame}

\hypertarget{simulation-studies}{%
\section{Simulation Studies}\label{simulation-studies}}

\begin{frame}{Exploring Smoothing Splines, Penalized Regression, Boosting, and SVM}
\protect\hypertarget{exploring-smoothing-splines-penalized-regression-boosting-and-svm}{}
\begin{itemize}
\item
  Draw \(X\) from a (univariate or multivariate) uniform distribution and the error from a normal distribution (or other distributions) with the same variance.
\item
  Overlay the cross-validated MSE curve obtained from a separate training set.
\item
  Overlay the test MSE curve from an independent test set

  \begin{itemize}
  \tightlist
  \item
    This is ``bogus'' since in real scenarios We wouldn't have been able to find the MSE since the response values are typically unknown.
  \item
    We do this to get a sense of how close we would have been to the (bogus, but practical) optimum for that particular test set if we used the hyperparameter from CV.
  \end{itemize}
\end{itemize}
\end{frame}

\begin{frame}{Smoothing Splines}
\protect\hypertarget{smoothing-splines}{}
\(f_1\)

\begin{figure}

{\centering \includegraphics[width=0.45\linewidth,height=1\textheight]{splines_1_f_1} \includegraphics[width=0.45\linewidth,height=1\textheight]{splines_2_f_1} 

}

\caption{Normal Errors (Left) and Double Expenential Errors (Right)}\label{fig:splines-plot-1}
\end{figure}
\end{frame}

\begin{frame}{Smoothing Splines}
\protect\hypertarget{smoothing-splines-1}{}
\(f_2\)

\begin{figure}

{\centering \includegraphics[width=0.45\linewidth,height=1\textheight]{splines_1_f_2} \includegraphics[width=0.45\linewidth,height=1\textheight]{splines_2_f_2} 

}

\caption{Normal Errors (Left) and Double Expenential Errors (Right)}\label{fig:splines-plot-2}
\end{figure}
\end{frame}

\begin{frame}{Smoothing Splines}
\protect\hypertarget{smoothing-splines-2}{}
\(f_3\)

\begin{figure}

{\centering \includegraphics[width=0.45\linewidth,height=1\textheight]{splines_1_f_3} \includegraphics[width=0.45\linewidth,height=1\textheight]{splines_2_f_3} 

}

\caption{Normal Errors (Left) and Double Expenential Errors (Right)}\label{fig:splines-plot-3}
\end{figure}
\end{frame}

\begin{frame}{Remarks}
\protect\hypertarget{remarks}{}
\end{frame}

\begin{frame}{Penalized Regression \(f_1\)}
\protect\hypertarget{penalized-regression-f_1}{}
\begin{figure}

{\centering \includegraphics[width=0.45\linewidth,height=1\textheight]{Ridge_f_1} \includegraphics[width=0.45\linewidth,height=1\textheight]{Lasso_f_1} 

}

\caption{RIdge (Left) and Lasso (Right)}\label{fig:penalized-1}
\end{figure}
\end{frame}

\begin{frame}{Penalized Regression \(f_2\)}
\protect\hypertarget{penalized-regression-f_2}{}
\begin{figure}

{\centering \includegraphics[width=0.45\linewidth,height=1\textheight]{Ridge_f_2} \includegraphics[width=0.45\linewidth,height=1\textheight]{Lasso_f_2} 

}

\caption{RIdge (Left) and Lasso (Right)}\label{fig:penalized-2}
\end{figure}
\end{frame}

\begin{frame}{Penalized Regression \(f_3\)}
\protect\hypertarget{penalized-regression-f_3}{}
\begin{figure}

{\centering \includegraphics[width=0.45\linewidth,height=1\textheight]{Ridge_f_3} \includegraphics[width=0.45\linewidth,height=1\textheight]{Lasso_f_3} 

}

\caption{RIdge (Left) and Lasso (Right)}\label{fig:penalized-3}
\end{figure}
\end{frame}

\begin{frame}{Penalized Regression \(f_4\)}
\protect\hypertarget{penalized-regression-f_4}{}
\begin{figure}

{\centering \includegraphics[width=0.45\linewidth,height=1\textheight]{Ridge_f_4} \includegraphics[width=0.45\linewidth,height=1\textheight]{Lasso_f_4} 

}

\caption{RIdge (Left) and Lasso (Right)}\label{fig:penalized-4}
\end{figure}
\end{frame}

\begin{frame}{Remarks}
\protect\hypertarget{remarks-1}{}
\end{frame}

\begin{frame}{Boosting \(f_1\), \(f_2\)}
\protect\hypertarget{boosting-f_1-f_2}{}
\begin{figure}

{\centering \includegraphics[width=0.45\linewidth,height=0.7\textheight]{Boosting_f_1} \includegraphics[width=0.45\linewidth,height=0.7\textheight]{Boosting_f_2} 

}

\caption{Boosting}\label{fig:boosting-1}
\end{figure}
\end{frame}

\begin{frame}{Boosting \(f_3\), \(f_4\)}
\protect\hypertarget{boosting-f_3-f_4}{}
\begin{figure}

{\centering \includegraphics[width=0.45\linewidth,height=0.7\textheight]{Boosting_f_3} \includegraphics[width=0.45\linewidth,height=0.7\textheight]{Boosting_f_4} 

}

\caption{Boosting}\label{fig:boosting-2}
\end{figure}
\end{frame}

\begin{frame}{Remarks}
\protect\hypertarget{remarks-2}{}
\end{frame}

\begin{frame}{SVM \(f_1\), \(f_2\)}
\protect\hypertarget{svm-f_1-f_2}{}
\begin{figure}

{\centering \includegraphics[width=0.45\linewidth,height=0.7\textheight]{svm_f_1} \includegraphics[width=0.45\linewidth,height=0.7\textheight]{svm_f_2} 

}

\caption{SVM}\label{fig:svm-1}
\end{figure}
\end{frame}

\begin{frame}{SVM \(f_3\), \(f_4\)}
\protect\hypertarget{svm-f_3-f_4}{}
\begin{figure}

{\centering \includegraphics[width=0.45\linewidth,height=0.7\textheight]{svm_f_3} \includegraphics[width=0.45\linewidth,height=0.7\textheight]{svm_f_4} 

}

\caption{SVM}\label{fig:svm-2}
\end{figure}
\end{frame}

\begin{frame}{Remarks}
\protect\hypertarget{remarks-3}{}
\end{frame}

\begin{frame}
\begin{figure}
\hypertarget{id}{%
\centering
\includegraphics[width=0.7\textwidth,height=0.7\textheight]{./tradeoff.jpg}
\caption{It's all about tradeoffs!}\label{id}
}
\end{figure}
\end{frame}

\hypertarget{q-a}{%
\section{Q \& A}\label{q-a}}

\begin{frame}{Thank You!}
\protect\hypertarget{thank-you}{}
\end{frame}

\end{document}
